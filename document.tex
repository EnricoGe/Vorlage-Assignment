% Vorgaben Assignment aus Studienheft SQL03
% Formatvorgaben fuer den Text
% Umfang: 8 - 10 Seiten (inkl. Abbildungen und Tabellen, aber ohne Deckblatt,
% Gliederung und Literaturverzeichnis, Eidesstattliche Erklaerung
% Zeilenabstand: 1,5
% Schriftart: frei
% Schriftgrad: 12 pt
% Variablen, physikalische Groessen und Funktionszeichen werden kursiv gedruckt.
% Korrekturrand: links: 4,5 cm, rechts 2,0 cm, oben und unten jeweils 3,0 cm
% Deckblatt: (Adresse, AKAD-E-Mail-Adresse, Immatrikulationsnummer, Modul-
% bezeichnung, Thema, Datum, Felder f�r Korrektor)
% Gliederung (1 Seite)
% Literaturverzeichnis (3 - 5 Literaturquellen  z. B. Lehrbuecher, aktuelle Fachartikel recherchieren)
% Eidesstattliche Erklaerung (unterschrieben und fest eingebunden)
% Bearbeitungsdauer: 2 Monate

\documentclass[
 paper=a4,      %Papierformat
 fontsize=12pt, %Schriftgr��e
 twoside=false, %einseitiger Druck, ansonsten Leerseiten bei Verzeichnissen
 %BCOR=16mm     %Bindekorrektur
 toc=listof,    %f�gt Abbildungs- & Tabellenverzeichnis ins Inhaltsverzeichnis ein
 %draft,			% Boxen anzeigen..
 fleqn			%Ausrichtung linksb�ndig (Formeln)
]{scrartcl}

%######################################
%### Schriftart und Rechtschreibung ###
%######################################
 
\usepackage[ngerman]{babel}       	% neue deutsche Rechtschreibung
 %\usepackage[utf8]{inputenc}      	% Eingabe ist UTF-8 - nicht bei XeLaTex
  
  
\usepackage[onehalfspacing]{setspace}    % Zeilenabstand definieren (1.5 zeilig)

\usepackage{mathptmx}	 % Times
\usepackage{fontspec}				% notwendig f�r Open Type Schriftarten
% \usepackage[T1]{fontenc}        	% bessere und richtige Schriftausgabe
 \setmainfont{Arial}

\addtokomafont{section}{\setmainfont{Arial}}
\addtokomafont{subsection}{\setmainfont{Arial}}
\addtokomafont{subsubsection}{\setmainfont{Arial}}
\addtokomafont{sectionentry}{\setmainfont{Arial}}

%#################################
%### Formatierung Seitenr�nder ###
%#################################
 
\usepackage{geometry}
 \geometry
 {  a4paper,
	left=45mm,     %linker Seitenrand
	right=20mm,    %rechter Seitenrand
 	top=30mm,      %oben der Abstand
 	bottom=30mm    %Unten der Abstand
 }

%####################
%### Mathe-Pakete ###
%####################
 
%mehr unter: http://www.golatex.de/tutorials-dokumentationen-pakete-fuer-mathematik-mit-latex-t2017.html
 \usepackage{amsmath}    %Grundpaket f�r mathematische Formeln
 %\usepackage{amsthm}    %Paket f�r Theoreme
 %\usepackage{amscd}     %Paket f�r Diagramme
 %\usepackage{amsfont}   %Schriftart

%#####################
%### Zusatz-Pakete ###
%#####################
 
 \usepackage{textcomp}                  % beinhaltet Sonderzeichen (�������...)
 \usepackage{eurosym}                   % offizielles Eurosymbol \euro
 \usepackage[printonlyused]{acronym}    % f�r das Abk�rzungsverzeichnis
 \usepackage{graphicx}                  % f�r externe Grafiken/Bilder
 \usepackage{epsfig}					% Einbinden von eps Grafiken
 \usepackage{fancyhdr}					% Gestaltung von Kopf- Fu�zeilen
 \usepackage[nottoc]{tocbibind} 		% Anzeigen des Literaturverzeichnisses im TOC
 \usepackage{float} 					% Notwendig fuer figure[h]
 %\usepackage{tocbasic}					% Verzeichnisse
 %\usepackage{listings}
 \usepackage{setspace}
 
 \usepackage{natbib}		% Literaturverzeichnis
 
%###################################
%### Konfiguration Verzeichnisse ###
%###################################
\newif\iflistoffigures
\newif\iflistoftables
\newif\ifacronym

\DeclareNewTOC[%
  type=formel,
  name={Formel},%
  hang=3em,%
  listname={Formelverzeichnis}
]{for}

\newcommand*{\formelentry}[1]{%
  \addcontentsline{for}{formel}{\protect\numberline{\theequation}#1}%
}

% Text unter Abbildungen
\renewcommand{\bflabel}[1]{\normalfont{\normalsize{#1}}\hfill}


%Titel
\newcommand*{\Titel}{Titel auf dem Deckblatt}

%Seitentitel
\newcommand*{\STitel}{Seiten�berschrift}

%Betreff
\newcommand*{\Betreff}{Assignment im Modul XXX99} 

\newcommand*{\Untertitel}{Eventuell ein Untertitel}

%Betreuer
\newcommand*{\Betreuer}{Betreuer: Max Mustermann} 

%Vor- und Nachname
\newcommand*{\Name}{Max Mustermann}

%Stra�e und Hausnummer
\newcommand*{\Strasse}{Musterstr. 123} 

%Plz und Ort
\newcommand*{\PlzOrt}{12345 Musterstadt} 

%Immatrikulationsnummer
\newcommand*{\Immatrikulationsnummer}{1 234 567}

%Email 
\newcommand*{\Email}{max.mustermann@akad.de} 

%Hochschule 
\newcommand*{\Hochschule}{AKAD Hochschule Stuttgart} 


% Verzeichnisse (Wenn nicht ben�tigt, Zeile mit % auskommentieren oder l�schen

% Abbildungsverzeichnis 
\listoffigurestrue

% Tabellenverzeichnis
%\listoftablestrue

% Abk�rzungsverzeichnis
\acronymtrue

% Formelverzeichnis
%\listofformelntrue

%###################################
%### Konfiguration PDF Erzeugung ###
%###################################

\usepackage[pdfa,xetex]{hyperref}	% PDF/A-konforme Flags erw�nscht

\hypersetup{
	pdfauthor={Enrico George},
	pdfcreator={},	% \XeTeX{} nicht m�glich
	pdfdisplaydoctitle,		% Dokument-Titel anzeigen
	pdfkeywords={akad, assignment, meta, information, pdf, hyperref, latex},
	pdftitle={\STitel},
	unicode,
	colorlinks=false,
	pdfborder={0 0 0},
}

\pagenumbering{Roman}

\makeatletter
\newcommand{\bibstyle@dinat}{\bibpunct{[}{]}{;}{a}{,}{,~}}
\makeatother

%----------------------------------------------------------------------------------------
% Worttrennungen

%\hyphenation{ }
%----------------------------------------------------------------------------------------
\addto{\captionsngerman}{%
\renewcommand{\refname}{Literaturverzeichnis}
}


\begin{document}

%%%%%%%%%%%%%%%%%%%%%%%%%%%%%%%%%%%%%%%%%%%%%%%%%%%%%%%%%%%%%
%% Deckblatt
%%%%%%%%%%%%%%%%%%%%%%%%%%%%%%%%%%%%%%%%%%%%%%%%%%%%%%%%%%%%%
%%
%% ACHTUNG: Sie ben�tigen ein Hauptdokument, um diese Datei
%%          benutzen zu k�nnen. Verwenden Sie im Hauptdokument
%%          den Befehl "\input{dateiname}", um diese
%%          Datei einzubinden.
%%

\thispagestyle{empty}
\begin{titlepage}
\parskip=1em
\parindent=0cm

\Name \\ \Strasse \\ \PlzOrt \\
\href{mailto:\Email}{\Email} \\ \\
Immatrikulationsnummer: \Immatrikulationsnummer \\
\Hochschule \\
\vspace{2cm}

\Huge{\Titel}
\vspace{1cm}
\onehalfspacing

\Large{\Betreff} \\
\large{\Untertitel}

\vspace{1cm}
\normalsize
\Betreuer \\ \\
\today
\vfill
\includegraphics[scale=0.35]{Abbildungen/akad_logo.png} \\
\Hochschule
\end{titlepage}

\clearpage
\normalsize

\setcounter{page}{1}
% Verzeichnisse werden mit einzeiligem Abstand gesetzt
\begin{spacing}{1.0} 
\newpage

% Inhaltsverzeichnis
\tableofcontents                                    %Inhaltsverzeichnis
\newpage

% Abbildungsverzeichnis
\iflistoffigures
\listoffigures 
\newpage
\fi

% Tabellenverzeichnis
\iflistoftables
\listoftables 
\newpage
\fi

% Abk�rzungsverzeichnis
\ifacronym
\section*{Abk�rzungsverzeichnis}
\addcontentsline{toc}{section}{Abk�rzungsverzeichnis} 
\begin{acronym}[BilMoG] %hier lie l�ngste Abk�rzung rein...
	\acro{BilMoG} {Bilanzrechtsmodernisierungsgesetz}
	\acro{HGB} {Handelsgesetzbuch}
	\acro{GKV} {Gemeinkostenverfahren}
	\acro{GuV} {Gewinn- und Verlustrechnung}
	\acro{PublG} {Publizit�tsgesetz }
	\acro{UKV} {Umsatzkostenverfahren}
\end{acronym}

\fi

% Formelverzeichnis
%\listofformels
%\addcontentsline{toc}{section}{Formelverzeichnis} 
%\newpage

\end{spacing} 

\clearpage

\newcounter{romanPagenumber} 
\setcounter{romanPagenumber}{\value{page}} % Roemische Seitenanzahl speichern.

\pagestyle{fancy}
	\fancyhead{}
	\fancyhead[L]{\textsc{\STitel}} % Textsc --> Kapit�lchen
	\fancyhead[R]{\thepage}
	\fancyfoot{}

% Alles ins Verzeichnis, auch wenn nicht zitiert...
%\nocite{*} 

\pagenumbering{arabic}

\begin{spacing}{1.5} % Zeilenabstand: 1,5 fuer den Textteil

\section{Einleitung}
\subsection{Schritt 1}

\subsection{Schritt 2}


\newpage
\section{Hauptteil}
\subsection{Schritt A}
\subsection{Schritt B}



\input{schluss}

\end{spacing}

\clearpage

% Literaturverzeichniss - Ab hier wieder Roemische Seitenzahlen
\pagestyle{plain}
\pagenumbering{Roman}
\setcounter{page}{\theromanPagenumber}

\bibliographystyle{dinat}

\bibliography{literatur}
\onehalfspacing
\clearpage
% Anhang...
%\include{anhang}
%\clearpage

\pagestyle{empty} 
\thispagestyle{empty}

\begin{center}
{\Large Eidesstattliche Erkl"arung}
\vspace*{4cm}\end{center}
\noindent
Ich versichere, dass ich das beiliegende Assignment selbstst"andig verfasst, keine anderen als die angegebenen Quellen und Hilfsmittel benutzt sowie alle w"ortlich oder sinngem"a"s "ubernommenen Stellen in der Arbeit gekennzeichnet habe. 
\vspace{3cm}

\hspace{-0.8cm}
\rule[0.5ex]{6.5cm}{1pt}
\hspace{1.3cm}
\rule[0.5ex]{6.5cm}{1pt} \\
(Datum, Ort)
\hspace{6.0cm}(Unterschrift)

%\input{drucktest}

\end{document}

