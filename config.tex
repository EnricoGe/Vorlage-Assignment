%######################################
%### Schriftart und Rechtschreibung ###
%######################################
 
\usepackage[ngerman]{babel}       	% neue deutsche Rechtschreibung
 %\usepackage[utf8]{inputenc}      	% Eingabe ist UTF-8 - nicht bei XeLaTex
  
  
\usepackage[onehalfspacing]{setspace}    % Zeilenabstand definieren (1.5 zeilig)

\usepackage{mathptmx}	 % Times
\usepackage{fontspec}				% notwendig f�r Open Type Schriftarten
% \usepackage[T1]{fontenc}        	% bessere und richtige Schriftausgabe
 \setmainfont{Arial}

\addtokomafont{section}{\setmainfont{Arial}}
\addtokomafont{subsection}{\setmainfont{Arial}}
\addtokomafont{subsubsection}{\setmainfont{Arial}}
\addtokomafont{sectionentry}{\setmainfont{Arial}}

%#################################
%### Formatierung Seitenr�nder ###
%#################################
 
\usepackage{geometry}
 \geometry
 {  a4paper,
	left=45mm,     %linker Seitenrand
	right=20mm,    %rechter Seitenrand
 	top=30mm,      %oben der Abstand
 	bottom=30mm    %Unten der Abstand
 }

%####################
%### Mathe-Pakete ###
%####################
 
%mehr unter: http://www.golatex.de/tutorials-dokumentationen-pakete-fuer-mathematik-mit-latex-t2017.html
 \usepackage{amsmath}    %Grundpaket f�r mathematische Formeln
 %\usepackage{amsthm}    %Paket f�r Theoreme
 %\usepackage{amscd}     %Paket f�r Diagramme
 %\usepackage{amsfont}   %Schriftart

%#####################
%### Zusatz-Pakete ###
%#####################
 
 \usepackage{textcomp}                  % beinhaltet Sonderzeichen (�������...)
 \usepackage{eurosym}                   % offizielles Eurosymbol \euro
 \usepackage[printonlyused]{acronym}    % f�r das Abk�rzungsverzeichnis
 \usepackage{graphicx}                  % f�r externe Grafiken/Bilder
 \usepackage{epsfig}					% Einbinden von eps Grafiken
 \usepackage{fancyhdr}					% Gestaltung von Kopf- Fu�zeilen
 \usepackage[nottoc]{tocbibind} 		% Anzeigen des Literaturverzeichnisses im TOC
 \usepackage{float} 					% Notwendig fuer figure[h]
 %\usepackage{tocbasic}					% Verzeichnisse
 %\usepackage{listings}
 \usepackage{setspace}
 
 \usepackage{natbib}		% Literaturverzeichnis
 
%###################################
%### Konfiguration Verzeichnisse ###
%###################################
\newif\iflistoffigures
\newif\iflistoftables
\newif\ifacronym

\DeclareNewTOC[%
  type=formel,
  name={Formel},%
  hang=3em,%
  listname={Formelverzeichnis}
]{for}

\newcommand*{\formelentry}[1]{%
  \addcontentsline{for}{formel}{\protect\numberline{\theequation}#1}%
}

% Text unter Abbildungen
\renewcommand{\bflabel}[1]{\normalfont{\normalsize{#1}}\hfill}

